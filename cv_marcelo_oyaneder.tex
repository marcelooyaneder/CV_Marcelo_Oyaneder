%%%%%%%%%%%%%%%%%
% This is an sample CV template created using altacv.cls
% (v1.1.5, 1 December 2018) written by LianTze Lim (liantze@gmail.com). Now compiles with pdfLaTeX, XeLaTeX and LuaLaTeX.
%
%% It may be distributed and/or modified under the
%% conditions of the LaTeX Project Public License, either version 1.3
%% of this license or (at your option) any later version.
%% The latest version of this license is in
%%    http://www.latex-project.org/lppl.txt
%% and version 1.3 or later is part of all distributions of LaTeX
%% version 2003/12/01 or later.
%%%%%%%%%%%%%%%%

%% If you need to pass whatever options to xcolor
\PassOptionsToPackage{dvipsnames}{xcolor}


%% If you are using \orcid or academicons
%% icons, make sure you have the academicons
%% option here, and compile with XeLaTeX
%% or LuaLaTeX.
%\documentclass[10pt,a4paper,academicons]{altacv}

%% Use the "normalphoto" option if you want a normal photo instead of cropped to a circle
% \documentclass[10pt,a4paper,normalphoto]{altacv}

\documentclass[10pt,letter,ragged2e]{altacv}

\usepackage{hyperref}

%% AltaCV uses the fontawesome and academicon fonts
%% and packages.
%% See texdoc.net/pkg/fontawecome and http://texdoc.net/pkg/academicons for full list of symbols. You MUST compile with XeLaTeX or LuaLaTeX if you want to use academicons.

% Change the page layout if you need to
\geometry{left=1cm,right=9cm,marginparwidth=6.8cm,marginparsep=1.2cm,top=1.25cm,bottom=1.25cm}

% Change the font if you want to, depending on whether
% you're using pdflatex or xelatex/lualatex
\ifxetexorluatex
  % If using xelatex or lualatex:
  \usepackage{Arial}
\else
  % If using pdflatex:
  \usepackage[utf8]{inputenc}
  \usepackage[T1]{fontenc}
  \usepackage[default]{lato}
\fi

% Change the colours if you want to
\definecolor{Mulberry}{HTML}{72243D}
\definecolor{SlateGrey}{HTML}{2E2E2E}
\definecolor{LightGrey}{HTML}{666666}
\colorlet{heading}{Sepia}
\colorlet{accent}{Mulberry}
\colorlet{emphasis}{SlateGrey}
\colorlet{body}{LightGrey}

% Change the bullets for itemize and rating marker
% for \cvskill if you want to
\renewcommand{\itemmarker}{{\small\textbullet}}
\renewcommand{\ratingmarker}{\faCircle}

%% sample.bib contains your publications
\addbibresource{sample.bib}

\begin{document}
\name{Marcelo Oyaneder Labarca}
\tagline{Lic. en Cs. de la Ingeniería}
%\photo{2.8cm}{Picture}
\personalinfo{%
  % Not all of these are required!
  % You can add your own with \printinfo{symbol}{detail}
  \href{mailto:marcelo.oyaneder@usach.cl}{\email{marcelo.oyaneder@usach.cl}} 
  \phone{+56 9 5892 8256}\\
%  \mailaddress{Address, Street, 00000 County}
  \location{Jose ghiardo 0210 C/38, La granja, Santiago}
%  \homepage{www.homepage.com/}
%  \twitter{@twitterhandle}
\href{linkedin.com/in/marcelooyanederlabarca}{\linkedin{linkedin.com/in/marcelooyanederlabarca}}
\href{https://github.com/marcelooyaneder/}{\github{github.com/marcelooyaneder/}}
  %% You MUST add the academicons option to \documentclass, then compile with LuaLaTeX or XeLaTeX, if you want to use \orcid or other academicons commands.
  % \orcid{orcid.org/0000-0000-0000-0000}
}

%% Make the header extend all the way to the right, if you want.
\begin{fullwidth}
\makecvheader
\end{fullwidth}

%% Depending on your tastes, you may want to make fonts of itemize environments slightly smaller
% \AtBeginEnvironment{itemize}{\small}

%% Provide the file name containing the sidebar contents as an optional parameter to \cvsection.
%% You can always just use \marginpar{...} if you do
%% not need to align the top of the contents to any
%% \cvsection title in the "main" bar.
\cvsection[page1sidebar]{Experiencia}

\cvevent{Informática de la Biodiversidad}{Laboratorio biología de plantas}{2019-Actualidad}{Universidad de Chile, Antumapu}
\begin{itemize}
\item{Desarrollo de software para la automatización y análisis de bases de datos basadas en el estándar Darwin core.}
\item{Desarrollo de script para procesar georeferenciación en Batch mediante Georeferencing Calculator}
\end{itemize}

\divider

\cvevent{Ayudante}{Departamento de Ingeniería Química}{2019-Actualidad}{Universidad de Santiago de Chile}
\begin{itemize}
\item{Ayudante de laboratorio en el curso transferencia de calor.}
\item{Ayudante de cátedra en el curso Calculo de Procesos. }
\end{itemize}

\divider

\cvevent{Adaptador de textos}{Departamento de formación integral e inclusión}{2019-2020}{Universidad de Santiago de Chile}
\begin{itemize}
\item Adaptación de textos para estudiantes en situación de discapacidad.
\end{itemize}

\divider

\cvevent{Clases particulares}{Independiente}{2012-Actualidad}{}
Realización de clases particulares a estudiantes pertenecientes al modulo básico de ingeniería como a escolares.

\cvsection{Estudios}

\cvevent{Educación superior}{Universidad de Santiago de Chile}{2015-Actualidad}{Est. central, RM, Chile}
\begin{itemize}
\item Ingeniería civil química
\end{itemize}

\divider

\cvevent{Educación básica y media}{Instituto Nacional Gral. Jose Miguel Carrera}{2008-2014}{Santiago, RM, Chile}
\begin{itemize}
\item 7$^\circ$ básico a 4$^\circ$ medio
\end{itemize}

\divider

\cvevent{Educación básica}{Colegio Tomas Moro}{2002-2007}{San miguel, RM, Chile}
\begin{itemize}
\item 1$^\circ$ a 6$^\circ$ básico
\end{itemize}

\medskip

%\clearpage
%\cvsection[page2sidebar]{Publications}

\nocite{*}

\printbibliography[heading=pubtype,title={\printinfo{\faBook}{Books}},type=book]

\divider

\printbibliography[heading=pubtype,title={\printinfo{\faFileTextO}{Journal Articles}},type=article]

%\divider

\printbibliography[heading=pubtype,title={\printinfo{\faGroup}{Conference Proceedings}},type=inproceedings]

%% If the NEXT page doesn't start with a \cvsection but you'd
%% still like to add a sidebar, then use this command on THIS
%% page to add it. The optional argument lets you pull up the
%% sidebar a bit so that it looks aligned with the top of the
%% main column.
% \addnextpagesidebar[-1ex]{page3sidebar}


\end{document}

%%%%%%%%%%%%%%%%%
% This is an sample CV template created using altacv.cls
% (v1.1.5, 1 December 2018) written by LianTze Lim (liantze@gmail.com). Now compiles with pdfLaTeX, XeLaTeX and LuaLaTeX.
%
%% It may be distributed and/or modified under the
%% conditions of the LaTeX Project Public License, either version 1.3
%% of this license or (at your option) any later version.
%% The latest version of this license is in
%%    http://www.latex-project.org/lppl.txt
%% and version 1.3 or later is part of all distributions of LaTeX
%% version 2003/12/01 or later.
%%%%%%%%%%%%%%%%

%% If you need to pass whatever options to xcolor
\PassOptionsToPackage{dvipsnames}{xcolor}


%% If you are using \orcid or academicons
%% icons, make sure you have the academicons
%% option here, and compile with XeLaTeX
%% or LuaLaTeX.
%\documentclass[10pt,a4paper,academicons]{altacv}

%% Use the "normalphoto" option if you want a normal photo instead of cropped to a circle
% \documentclass[10pt,a4paper,normalphoto]{altacv}

\documentclass[10pt,letter,ragged2e]{altacv}

\usepackage{hyperref}

%% AltaCV uses the fontawesome and academicon fonts
%% and packages.
%% See texdoc.net/pkg/fontawecome and http://texdoc.net/pkg/academicons for full list of symbols. You MUST compile with XeLaTeX or LuaLaTeX if you want to use academicons.

% Change the page layout if you need to
\geometry{left=1cm,right=9cm,marginparwidth=6.8cm,marginparsep=1.2cm,top=1.25cm,bottom=1.25cm}

% Change the font if you want to, depending on whether
% you're using pdflatex or xelatex/lualatex
\ifxetexorluatex
  % If using xelatex or lualatex:
  \usepackage{Arial}
\else
  % If using pdflatex:
  \usepackage[utf8]{inputenc}
  \usepackage[T1]{fontenc}
  \usepackage[default]{lato}
\fi

% Change the colours if you want to
\definecolor{Mulberry}{HTML}{72243D}
\definecolor{SlateGrey}{HTML}{2E2E2E}
\definecolor{LightGrey}{HTML}{666666}
\colorlet{heading}{Sepia}
\colorlet{accent}{Mulberry}
\colorlet{emphasis}{SlateGrey}
\colorlet{body}{LightGrey}

% Change the bullets for itemize and rating marker
% for \cvskill if you want to
\renewcommand{\itemmarker}{{\small\textbullet}}
\renewcommand{\ratingmarker}{\faCircle}

%% sample.bib contains your publications
\addbibresource{sample.bib}

\begin{document}
\name{Marcelo Oyaneder Labarca}
\tagline{Ingeniero Civil Químico}
%\photo{2.8cm}{Picture}
\personalinfo{%
  % Not all of these are required!
  % You can add your own with \printinfo{symbol}{detail}
  \href{mailto:marcelo.oyaneder@usach.cl}{\email{marcelo.oyaneder@usach.cl}} 
  \phone{+56 9 5892 8256}\\
%  \mailaddress{Address, Street, 00000 County}
  \location{Concha \#8991 Depto. 530-G, La Cisterna, Santiago}
%  \homepage{www.homepage.com/}
%  \twitter{@twitterhandle}
\href{https://www.linkedin.com/in/marcelooyanederlabarca/}{\linkedin{linkedin.com/in/marcelooyanederlabarca}}
\href{https://github.com/marcelooyaneder/}{\github{github.com/marcelooyaneder/}}
  %% You MUST add the academicons option to \documentclass, then compile with LuaLaTeX or XeLaTeX, if you want to use \orcid or other academicons commands.
  % \orcid{orcid.org/0000-0000-0000-0000}
}

%% Make the header extend all the way to the right, if you want.
\begin{fullwidth}
\makecvheader
\end{fullwidth}

%% Depending on your tastes, you may want to make fonts of itemize environments slightly smaller
\AtBeginEnvironment{itemize}{\small}

%% Provide the file name containing the sidebar contents as an optional parameter to \cvsection.
%% You can always just use \marginpar{...} if you do
%% not need to align the top of the contents to any
%% \cvsection title in the "main" bar.
\cvsection[page1sidebar]{Experiencia}
\cvevent{Profesor por hora}{Universidad de Santiago de Chile - Departamento de Ingeniería Química}{2022-Actualidad}{Santiago, Chile}
\begin{itemize}
	\item Docente a cargo del curso de Laboratorio de Transferencia de Calor, para estudiantes de Ing. Civil Química e Ing. en Biotecnologia , Dpto. de Ingeniería Química y Bioprocesos, Facultad de ingeniería.
\end{itemize}

\divider

\cvevent{Ingeniero de proyectos de ingeniería}{Partículas Ingeniería y Gestión Ambiental}{2021 - Actualidad}{Santiago, Chile}
\begin{itemize}
	\item Reportería automática y análisis de datos con énfasis en las componentes ambientales Agua y Aire. SMA, 2022-2023.	
	\item Implementación de una plataforma integrada para Pronóstico de calidad del aire. CODELCO DVEN, 2022.
	\item Servicios de análisis de Calidad del aire en bahía Quintero. CODELCO DVEN, 2022.
	\item Estudio de impactos y riesgos de salud por SO2. Anglo American, Los Bronces, 2022.
	\item Modelación y simulación CFD "Servicio de modelación de alternativas de control de emisiones fugitivas del dióxido de azufre". ENAMI, 2022.
	\item Servicio de Gestión Integral de la Calidad Aire. Minera Los Pelambres, 2021-2022.
  \item Pronóstico de la calidad del aire y meteorología para División El Soldado de Anglo American, 2021-2026.
  \item Modelación y simulación CFD, para proyecto CORFO AirCFD, 2021-2022.
  \item  Modelado y simulación CFD de distintas alternativas de sistema de ventilación y control de gases y polvo en área de fusión, ESCO Elecmetal S.A., 2021.
  \item Modelación CFD y diseño escenarios para mejorar transferencia de calor en horno de tratamiento térmico de piezas metálicas. ESCO Elecmetal S.A., 2021.
\end{itemize}

\divider

\cvevent{Informática de la Biodiversidad}{Laboratorio biología de plantas}{2019 - Actualidad}{Universidad de Chile, Antumapu}
\begin{itemize}
\item Catastrador vegetación natural y urbana.
\item{Desarrollo de software para la automatización y análisis de bases de datos basadas en el estándar Darwin core.}
\item{Desarrollo de Script para procesar georeferenciación en Batch mediante Georeferencing Calculator.}
\item Desarrollo de Script para la automatización de etiquetado en herbario (Proyecto Pudahuel).
\end{itemize}

\divider

\cvevent{Data Manager}{Millennium Institute Biodiversity of Antarctic and Subantarctic Ecosystems (BASE)}{Mayo, 2022 - Julio, 2022}{Universidad de Concepción}
\begin{itemize}
\item Generación de protocolos.
\item Gestión de bases de datos de biodiversidad.
\end{itemize}

\divider


\cvevent{Ayudante de profesor}{Departamento de Ingeniería Química}{2019 - 2021}{Universidad de Santiago de Chile}
\begin{itemize}
\item{Ayudante de laboratorio en el curso transferencia de calor.}
\item{Ayudante de cátedra en el curso Calculo de Procesos. }
\end{itemize}

\divider

%\cvevent{Adaptador de textos}{Departamento de formación integral e inclusión}{2019-2020}{Universidad de Santiago de Chile}
%\begin{itemize}
%\item Adaptación de textos para estudiantes en situación de discapacidad.
%\item Creación de formato estándar de adaptación basado en Lectura Fácil.
%\end{itemize}

\cvsection[page2sidebar]{Estudios}

\cvevent{Educación superior}{Universidad de Santiago de Chile}{2015 - 2021}{Est. central, RM, Chile}
\begin{itemize}
\item Titulo de tesis: Propuesta de desarrollo y mejora, en sistema de georreferenciación basado en el método del punto radio y en el estándar Darwin Core, para la recolección, reutilización y publicación de datos primarios de biodiversidad.
\end{itemize}

\divider

\cvevent{Educación básica y media}{Colegio Tomas Moro \& Instituto Nacional Gral. Jose Miguel Carrera}{2002 - 2014}{Santiago, RM, Chile}
\begin{itemize}
\item 7$^\circ$ básico a 4$^\circ$ Medio - Instituto Nacional Gral. Jose Miguel Carrera
\end{itemize}

\divider

\medskip

\cvsection{Presentaciones}

\cvevent{Movilización de datos abiertos de biodiversidad en Chile}{Flujo de datos primarios de biodiversidad en Chile, ¿Cómo aumentar la calidad de nuestros datos?}{Agosto, 2022}{SUBPESCA}
\divider

\cvevent{IV Seminario de Colecciones biologicas}{Flujo de información de biodiversidad en Chile: puntos focales para maximizar la calidad de los datos}{Diciembre, 2021}{}
\divider

\cvevent{Encuentro anual de monitoreo participativo de biodiversidad}{Análisis y potencialidades en el uso de los datos de biodiversidad de la plataforma de publicación y acceso abierto Global Biodiversity Information Facility –GBIF}{Agosto, 2021}{}

\newpage

\nocite{*}

\printbibliography[heading=pubtype,title={\printinfo{\faBook}{Publicaciones GBIF}},type=misc]

%\divider

\printbibliography[heading=pubtype,title={\printinfo{\faFileTextO}{Journal Articles}},type=article]

%\divider

\printbibliography[heading=pubtype,title={\printinfo{\faGroup}{Congresos}},type=inproceedings]

%% If the NEXT page doesn't start with a \cvsection but you'd
%% still like to add a sidebar, then use this command on THIS
%% page to add it. The optional argument lets you pull up the
%% sidebar a bit so that it looks aligned with the top of the
%% main column.
% \addnextpagesidebar[-1ex]{page3sidebar}


\end{document}
